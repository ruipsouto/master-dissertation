The amount of generated data is currently growing exponentially. As a result, relational database management systems continue to be the most widely used solution in business environments to support transactional data storage. Such systems store information in the form of tables, where data is stored in rows and columns.

The widespread use of relational databases is partly due to the use of declarative data query languages (i.e., \gls{sql}). The abstraction from the actual low-level details of the physical organization of data allows complex queries to be expressed concisely and straightforwardly while providing a lot of querying flexibility. For this reason, the user only needs to specify the form of the result to be obtained and not the procedure to achieve it.

The query processor is the module that reads the query and generates the low-level procedure that should be executed to obtain the result. It starts by transforming the user-declared query into a lower-level relational algebra representation that outlines an efficient execution plan.

An important aspect of query processing is optimization. During this process, the query is optimized, and the most efficient plan among the different possible strategies that can be used to process it is selected. Since users are not expected to write queries as efficiently as possible, it is up to the system to build an execution plan that minimizes their execution cost. Usually, this process includes three separate components: search space, cost model, and search strategy. The search space is the set of alternative execution plans that return the same result and is obtained by applying transformation rules at the relational algebra level. They vary in how the operations involved are conducted and how they are implemented, resulting in different performance levels. The cost model tries to estimate the cost of a single execution plan. Finally, the search strategy exploits the search space to choose the plan that maximizes performance based on cost model predictions.

%Although this entire process takes place in centralized and distributed environments, it is much more complex in distributed database systems as more factors interfere with performance. Additionally, the relations involved in a particular query may be fragmented and replicated, adding communication costs to the equation.

In recent years, we have witnessed the exponential growth in the study and development of machine learning applications in systems' optimization. It has opened different possibilities of applying machine learning techniques in the architecture of database systems, which are becoming increasingly complex and hard to tune. This dissertation studies the practicality and utility of sophisticated machine learning techniques regarding query optimization as a promising approach.