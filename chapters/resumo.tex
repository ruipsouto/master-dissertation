Os otimizadores de \textit{queries} são considerados um dos componentes de maior relevância e complexidade num sistema de gestão de bases de dados. No entanto, apesar de atualmente produzirem resultados quase ótimos, os otimizadores dependem do uso de estimativas estatísticas e de heurísticas para reduzir o espaço de procura de planos de execução alternativos para uma determinada \textit{query}. Como resultado, para \textit{queries} mais complexas, os erros podem crescer exponencialmente, o que geralmente se traduz em planos sub-ótimos, resultando num desempenho inferior ao ideal. Os recentes avanços nas técnicas de aprendizagem automática abriram novas oportunidades para muitos dos problemas existentes relacionados com otimização de sistemas.

Este documento propõe uma solução construída sobre o PostgreSQL que aprende a selecionar o conjunto mais eficiente de configurações do otimizador para uma determinada \textit{query}. Em vez de depender inteiramente de estimativas do otimizador para comparar planos de configurações diferentes, a solução baseia-se num algoritmo de seleção \textit{greedy} que suporta vários tipos de técnicas de modelagem preditiva, desde técnicas mais tradicionais a uma abordagem de \textit{deep learning}.

O sistema é avaliado experimentalmente com os \textit{workloads} \gls{tpch} e Join Ordering Benchmark para medir o custo e os benefícios de adicionar aprendizagem automática a otimizadores de \textit{queries} tradicionais.

\paragraph{Palavras-chave} Aprendizagem automática, otimização de \textit{queries}, \textit{tuning} de base de dados