This chapter introduces the relevant state of the art to this dissertation. Currently, relational database management systems rely on a vast number of query optimization techniques that were continuously developed and implemented into production systems over several decades.

First, an introduction to query processing is given, discussing the basic concepts and strategies for optimizing queries. It starts by characterizing the main components involved in the process and the standard algorithm used to enumerate semantically equivalent plan alternatives. Finally, dynamic query optimization techniques are discussed.

Moreover, an overview of the broad domain of machine learning is provided, describing its fundamental concepts, multiple paradigms, types of tasks, and different machine learning algorithms.

The final section narrows the gap between machine learning and optimization of database systems. It discusses possible courses of action and describes the relevant related work concerning machine learning techniques in database systems. In the end, based on the analysis of related work, a discussion around the limitations of the current solutions is presented as well as the guiding principles that are taken into consideration during the development of the proposed solution.